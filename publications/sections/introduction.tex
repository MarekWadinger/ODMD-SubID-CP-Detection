\subsection{Context and Motivation}
Many industrial systems are safety-critical, where process monitoring is essential to protect both profit and life. These dynamical systems operate at various operating points, often driven by control action to meet desired production goals. However, environmental fluctuations and varying component quality lead to unexpected and persistent changes in system behavior. These rare events can jeopardize optimal operations, accelerate wear and tear, and occasionally result in catastrophic consequences, such as equipment damage, production loss, or even human casualties.

Monitoring abrupt and gradual changes in system behavior is crucial for ensuring system reliability and safety, a task known as change-point detection (CPD). Traditional system monitoring methods, like Statistical Process Control (SPC), rely on the assumption that data are independent and identically distributed (i.i.d.), which is often not the case in industrial systems. Industrial process data are typically correlated and non-stationary, complicating the application of SPC methods. While SCADA-based systems do not rely on i.i.d. and use static thresholds to detect changes, they cannot adapt to the dynamic changes in system behavior due to aging or environmental shifts.

Conventionally, offline machine learning (ML) methods are employed to identify macro-scale events in complex, high-dimensional dynamical systems. These methods depend on historical data and require offline training to detect system behavior changes. Although supervised ML methods with annotated data offer high accuracy, they often fail in new contexts or when encountering unexpected data patterns. Moreover, these methods are impractical for existing industrial infrastructures where collecting data on a large scale is infeasible, and direct integration with ongoing data exchange services is necessary.

Indeed, industrial data are streamed and arrive at non-uniform rates, challenging methods dependent on uniform sampling. For instance, \citet{Liu2023} simultaneously detecting change points and anomalies by leveraging the rate of change, and \citet{Fathy2019} using cooperative adaptive filtering for change detection in wireless sensor networks both assume i.i.d. Therefore, these methods may not be suitable for non-uniform data streams typical in industrial settings.

Additionally, sequential data in industrial systems comprise distinct components: linear, seasonal, cyclic, regressions, interventions, and errors. Effective CPD methods must adapt to these changing conditions over the system's lifetime. Further consideration must be given to the type of change point we wish to detect. Variance change points affect more significant segments of time series, while additive change points are sudden pulses that die out quickly, and innovational change points are followed by gradual decay back to the original time series \citep{Srivastava2017}.

The key questions that arise are:
\begin{itemize}
	\item Can we detect changes in system behavior using streaming data?
	\item How can changes be detected in the presence of non-stationarity?
	\item Can we adapt to changing system behavior to maintain validity over the system's lifetime?
\end{itemize}

\subsection{Related Work}
Online CPD methods address these questions, which are central for real-time monitoring in safety-critical industrial systems. Unlike offline CPD methods, which typically provide robust detection with significant delay, online CPD offers real-time solutions essential for timely intervention and maintenance planning. Indeed, offline methods often retrospect the historical data and detect changes in the system behavior after the event has occurred. In some settings, this may be acceptable, and favorable properties of offline methods might be enjoyed. For example, \citet{Liu2022} developed a CPD framework using a dynamic Bayesian network model to capture causal relationships between variables, enhancing interpretability and credibility. However, in industrial environments, real-time monitoring is imperative to prevent production losses or catastrophic outcomes such as equipment failure.

% Refer to some successfully applied offline ML-based CPD methods (\citet{DeRyck2021}, ...)
Self-supervised approaches are frequently used in online CPD due to the impracticality of obtaining real-time ground truth annotations. However, they still rely on annotations to create a feedback loop. These may be available from other online subsystems that may provide labels for continually supervised approaches to CPD, such as~\cite{Korycki2021}. In most cases, the supervisory information is exploited directly from the raw unlabeled data, showing improved generalization abilities~\citep{Zhang2024}.  % Although the term self-supervision is mainly used in the machine learning community, we refer to any data-based system identification method as self-supervised, which relies on supervisory information solely generated by the data or the method itself.

\citet{Chu2022} propose a sequential nearest neighbor search for high-dimensional and non-Euclidean data streams. A stopping rule is proposed to alert detected CP as soon as it occurs while providing a maximum boundary on a number of false positives. Despite its innovation, the sensitivity of nearest neighbors methods to varying data densities and computational expense in high-dimensional spaces restricts its real-time applicability.

\citet{Gupta2022} proposed a three-phase architecture for real-time CPD using autoencoders (AE). However, the necessary preprocessing steps, such as shifting and scaling, require assumptions about data distribution that are often unknown in streaming scenarios. Additionally, recursive singular spectrum analysis, employed in the architecture, may impose significant computational overhead in the case of high-dimensional data.

\citet{Bao2024} proposes feature decomposition and contrastive learning (CoCPD) for industrial time series to detect both abrupt CPs and subtle changepoints. By isolating predictable components from residual terms, this method improves detection accuracy in detecting subtle changes. Contrastive learning methods rely on constructing negative samples to increase the energy of the change points and decrease the energy of the ongoing operation data. Nevertheless, this is one of the main bottlenecks of contrastive learning methods. Since changes are unpredictable events that differ in sources and nature, it is challenging to generate negative samples to capture this variability as the prior information on the magnitudes and timing distributions are unknown, and the space of negative samples is therefore unbounded.

Established statistical CPD methods promote interpretability while remaining highly competitive.~\citet{Rajaganapathy2022} introduced a Bayesian network-based CPD method, which is able to capture CPs characterized by step change leveraging causal relationships between the variables. Nevertheless, as we will explore soon, the change point may be characterized by a change in dynamics, which is better captured in the frequency domain rather than the time domain.

% Add transition

Another common practice in CPD is to compare past and future time series intervals using a dissimilarity measure, triggering alarms when intervals are sufficiently different. Statistical CPD methods usually compare the relative statistical differences between time intervals to identify change points (CPs). Temporal properties such as data distribution and time series models should be accurately modeled in advance to obtain more precise statistical metrics for evaluating interval homogeneity. Methods in this group define this dissimilarity measure based on the difference in distribution of the two intervals. For instance, CUSUM and related methods \citet{Ye2023} track changes in the parameter of a chosen distribution, and the generalized likelihood ratio (GLR) procedure \citet{Xie2013, Korycki2021} monitors the likelihood that both intervals are generated from the same distribution. Subspace-based methods measure the distance between subspaces spanned by the columns of an observability matrix \citet{Moskvina2003, Kawahara2007} or observe reconstruction error~\citep{DeRyck2021, Bao2024}. Here, we will use subspace-based methods as an umbrella term for decomposition and deep neural network methods, which rely on finding the low-dimensional description of the data.  % TODO: If references needed

\subsection{Subspace-based CPD}\label{sec:subspace-based-cpd}
The main advantages of subspace-based approaches include the absence of distributional assumptions and their ability to extract complex dynamical features from data efficiently. For example, \citet{Hirabaru2016} might be a powerful example of cost-effectiveness in high-dimensional systems. The authors find a 1D subspace within multidimensional data and apply efficient univariate CPD, expanding their applicability to multivariate scenarios. This operates under the same assumption as \citet{Fathy2019} that the measurements are closely related and enable 1D representation to capture the signal from the noise. Nevertheless, these approaches are not suitable for complex systems characterized by multiple weakly related quantities whose behavior cannot be captured solely by the largest eigenvalue.

While some ML-based transformers demonstrate the ability to adapt and generalize to new data while retaining useful information over prolonged deployments \citep{Corizzo2022}, they lack guarantees that the CPD score accurately reflects the actual dissimilarity between intervals. This issue, highlighted by \citet{DeRyck2021}, can lead to misjudgments about the severity of change points and result in poor decision-making. Additionally, subspace-based methods are sensitive to hyperparameter choices, often lacking informed guidance.

Subspace-based methods address these limitations by monitoring whether incoming data aligns with the null space of the reference state's observability matrix, effectively identifying new operating states \citep{Dohler2013, Ye2023}. \citet{Xie2013} leverage this principle in the MOUSSE (Multiscale Union of Subsets Model) algorithm, which tracks dynamic submanifolds in high-dimensional noisy data using a sequential generalized likelihood ratio procedure for CPD.

Numerous theoretical studies support the optimality of subspace-based methods in CPD. For instance, \citet{Ye2023} derive an exact subspace-CUSUM procedure and characterize average run length (ARL) and Type-II error probability using asymptotic random matrix theory, optimizing metrics such as expected detection delay (EDD). Similarly, \citet{Garreau2018} demonstrate that a kernel change-point algorithm can, with high probability, correctly identify the number and location of change points under well-chosen penalties and estimate the change-point location at the optimal rate.

Successful practical applications complement the solid theoretical foundation of these methods. For instance, \citet{Hosur2019} address the need for near real-time detection of changes in power systems' working conditions with a sequential detection algorithm based on stochastic subspace state-space identification, utilizing output-only covariance-based subspace identification with Hankel matrices. \citet{Dohler2013} propose a robust residual function for detecting changes in the eigenstructure of linear time-invariant systems for vibration monitoring. \citet{He2019} introduce ADMOST, an online subspace tracking framework similar to online SVD updated without increasing the rank, displaying its applicability to real-time UAV flight data, where anomaly detection and mitigation are required.

\subsection{DMD-based CPD}
In both autonomous and controlled dynamical systems, change points may be characterized by shifts in dynamics that are more effectively captured in the frequency domain rather than the time domain \citep{DeRyck2021, Gupta2022}. Addressing this issue requires decomposing a time series into its dominant frequency components, which are described by oscillations and magnitudes. For example, \citet{DeRyck2021} combined detection in both domains using two autoencoders in the TIRE method, leveraging discrete Fourier Transformation to extract spectral information. Similarly, \citet{Gupta2022} utilized recursive singular spectrum analysis in preprocessing within an autoencoder-based CPD framework to decompose time series into dominant frequency components. However, this approach requires retraining the model after each predicted change point, which is computationally expensive and unsuitable for real-time applications.

Dynamic Mode Decomposition (DMD) emerges as a suitable method for CPD in both the time and frequency domain. DMD is a data-driven technique that decomposes a time series into its dominant frequency components (modes), described by their oscillation and magnitudes. By approximating the dynamical system through a linear combination of these modes, DMD facilitates interpretable CPD concerning system dynamics. It allows for monitoring changes in spatial features and system dynamics and detecting changes arising from environmental factors.

Supporting this claim, \citet{Prasadan2020} applied DMD to a data matrix composed of linearly independent, additive mixtures of latent time series, focusing on missing data recovery. They demonstrated that hankelized DMD, a higher-lag extension of DMD, could unmix signals, revealing them better in noise and offering superior reconstruction compared to Principal Component Analysis (PCA) and Independent Components Analysis (ICA).

In another study, \citet{Srivastava2017} introduced an innovative offline algorithm leveraging DMD to detect variance change points iteratively. Their method integrates a data-driven dynamical system with a local adaptive window guided by a variance descriptor function, facilitating the identification of change points at various scales. By employing sequential hypothesis testing and a dynamic window mechanism, the method dynamically adjusts the window's location and size to detect changes in variance. This offline algorithm performs multiple passes over the data first to identify the longest stationary segments and then detect variance change points, making it unsuitable for real-time applications.

Similarly, \citet{Gottwald2020} focused on detecting transient dynamics and regime changes in time series using DMD. They argued that transitions between different dynamical regimes are often reflected in higher-dimensional space, followed by relaxation to lower-dimensional space. They proposed using the reconstruction error of DMD to monitor a time series' inability to resolve fast relaxation towards the attractor and the system dynamics' effective dimension.

\subsection{Research Objective and Contributions}
This paper proposes an online change-point detection (CPD) method based on truncated online Dynamic Mode Decomposition (DMD) with control. We leverage DMD's capability to decompose time series into dominant frequency components and incorporate control effects to adapt to changing system behaviors. The proposed method detects abrupt changes in system behavior, considering both the time and frequency domains. We demonstrate the effectiveness of this approach on real-world data streams, showing that it is highly competitive or superior to other general CPD methods in terms of detection accuracy on benchmark datasets.

The significance of this work is underscored in industrial settings where complex dynamical systems are challenging to describe, data arrive at non-uniform rates, and real-time assessment of changes is crucial to protecting both profit and life.

The main contributions of this paper are:
\begin{itemize}
	\item Formulation of a truncated version of online DMD with control for tracking system dynamics.
	\item Utilization of higher-order time-delay embeddings in streamed data to extract broad-band features.
	\item Demonstration that using the DMD improves the detection accuracy compared to SVD-based CPD methods.
	\item Analysis of the correspondence between increases in detection statistics and the actual dissimilarity of compared intervals.
	\item Validation of the proposed method's effectiveness on real-world data from a controlled dynamical system.
	\item Provision of intuitive guidelines for selecting hyperparameters for the proposed method.
\end{itemize}
