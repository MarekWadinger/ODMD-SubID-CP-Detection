In this paper, we proposed truncation of online Dynamic Mode Decomposition with control and examined its efficacy in online subspace-based change point detection tasks. The approximation of subspace over which a complex system (possibly a non-linear time-varying controlled system with delays) evolves is traced using time-delayed embeddings created directly from the system's input-output non i.i.d.~streaming data. DMD enables the decomposition of the system's dynamics into a set of modes that can be used to reconstruct signals from the data, which are subject to noise and carry information about abrupt changes. The similarity of the original data to its reconstruction is evaluated over two windows: reference and test. The former establishes base reconstruction error, and the latter, which includes the latest snapshots provided by the streaming service, is evaluated for the presence of a change point. The size of the test window defines the delay of the peak CPD statistics, as shown on the synthetic dataset, and defines the maximum delay of the alarm under the assumption that the error crosses the selected threshold. The tradeoff between detection speed and the number of false positives can be tuned by changing this parameter. Although setting generally applicable default values of the proposed method's hyperparameters is impossible, we establish intuitive guidelines for their selection. We also show that while computing CPD statistics on error ratio reveals minor change points close to the origin, error divergence can be used to acquire statistics proportional to the actual difference. In the case study displaying real-world examples of faulty HVAC operation detection in BESS, we observe that the height of difference of the errors is proportional to the distance of the faulty state from normal operation. This is crucial for assessing the severity of deviations in the operation of industrial systems, which is relevant in overall risk assessment. In contrast, the error ratio hints at potential precursors of the transition towards the faulty operation. The proposed method is highly competitive, as shown on two benchmark datasets of a simulated complex system and a real laboratory system.
